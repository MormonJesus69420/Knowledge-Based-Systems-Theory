% Article type supporting font formatting
\documentclass[a4,12pt]{extarticle}

% Define .tex file encoding
\usepackage[utf8]{inputenc}

% Norwegian language support
%\usepackage[norsk]{babel}     

% Indent first paragraph in section
\usepackage{indentfirst}

% Allows mathbb in tex file
\usepackage{amsfonts}    
     
% Margin defining package
\usepackage{geometry}         
\geometry{a4paper
  ,margin=1in
}

\usepackage[square,numbers]{natbib}

% For use of graphics in document
\usepackage{graphicx}         

% Allows multi-line comments in tex file
\usepackage{verbatim}         

% Allows math in tex file 
\usepackage{amsmath}          

% Allows math symbols in tex file
\usepackage{amssymb}          

% Allows use of physics shortcut functions
%\usepackage{physics}          

% Verbatim env with LaTeX commands
\usepackage{alltt}            

% Allows \begin{figure}[H]
\usepackage{float}            

% Necessary for defining colours
\usepackage{color}            
\definecolor{linkgreen}{rgb}{0,.5,0}
\definecolor{linkblue}{rgb}{0,0,.5}
\definecolor{linkred}{rgb}{.5,0,0}
\definecolor{blue}{rgb}{.13,.13,1}
\definecolor{green}{rgb}{0,.5,0}
\definecolor{red}{rgb}{.9,0,0}

% Hyperlinks in document
\usepackage{hyperref}  
\hypersetup{
  colorlinks=true,     % True for colored links
  linktoc=all,         % True for table of contents links
  linkcolor=linkblue,  % Colour for links
  urlcolor=linkgreen,  % Colour for URLs
  citecolor=linkred    % Colour for citations
}

% Listing package for code examples
\usepackage{listings}         
\lstset{
  language=C++,                % Set language to C++
  showspaces=false,            % Don't show space chars
  showtabs=false,              % Don't show tab chars
  breaklines=true,             % Break long lines of code
  showstringspaces=false,      % Don't show spaces in strings
  breakatwhitespace=true,      % Break at white space only
  commentstyle=\color{green},  % Set colour for comments
  keywordstyle=\color{blue},   % Set colours for keywords
  stringstyle=\color{red},     % Set colour for strings
  basicstyle=\ttfamily,        % Set basic style
  tabsize=2                    % Set tabsize
}

% Referencing, last for compatibility reasons
\usepackage[noabbrev]{cleveref}

% Command to set two lines under text
\newcommand{\uunderline}[1]{\underline{\underline{#1}}}

% Command to use integral with limits
\newcommand{\Int}{\int\limits}    

% Command to use double integral with limits
\newcommand{\IInt}{\iint\limits}  

% Command to use triple integral with limits
\newcommand{\IIInt}{\iiint\limits}

% Command removes section numbering
\newcommand{\mysection}[2]{   
\setcounter{section}{#1}
\section*{#2}
\addcontentsline{toc}{section}{#2}
}

% Command removes subsection numbering
\newcommand{\mysubsection}[2]{  
\setcounter{subsection}{#1}
\subsection*{#2}
\addcontentsline{toc}{subsection}{#2}
}

% Command removes subsubsection numbering
\newcommand{\mysubsubsection}[2]{ 
\setcounter{subsubsection}{#1}
\subsubsection*{#2}
\addcontentsline{toc}{subsubsection}{#2}
}

% Makes matrices look square-ish
\renewcommand*{\arraystretch}{1.5}

%%%%%%%%%%%%%%%%%%%%%%%%%%%%%%%%%%%%%%%
%%      Title, Author, and Date      %%
%%%%%%%%%%%%%%%%%%%%%%%%%%%%%%%%%%%%%%%
\title{Assignment 3 in Artificial Intelligence\\Spring 2018 }
\author{Daniel Aaron Salwerowicz}
\date{\today}

%%%%%%%%%%%%%%%%%%%%%%%%%%%%%%%%%%%%%%%
%%           Start document          %%
%%%%%%%%%%%%%%%%%%%%%%%%%%%%%%%%%%%%%%%
\begin{document}
  
%%%%%%%%%%%%%%%%%%%%%%%%%%%%%%%%%%%%%%%
%%   Create the main title section   %%
%%%%%%%%%%%%%%%%%%%%%%%%%%%%%%%%%%%%%%%
\maketitle

%%%%%%%%%%%%%%%%%%%%%%%%%%%%%%%%%%%%%%
%%  The main content of the report  %%
%%%%%%%%%%%%%%%%%%%%%%%%%%%%%%%%%%%%%%

\mysection{1}{Task 1.1}
\mysubsection{1}{Rules}
\begin{enumerate}
  \item \verb|IF   TS = Unchanged  AND  DJI = Up    THEN  Action = Buy|
  \item \verb|IF  DJI = Unchanged  AND   TS = Down  THEN  Action = Sell|
  \item \verb|IF  DJI = Down                        THEN  Action = Sell|
  \item \verb|IF   TS = Up                          THEN  Action = Buy|
  \item \verb|IF  DJI = Up                          THEN  Action = Wait|
  \item \verb|IF  DJI = Unchanged                   THEN  Action = Wait|
\end{enumerate}

\mysubsection{2}{Procedure}
\noindent
Sell:

DJI = Down (2+, 0-); New rule!

\noindent
Buy:

TS = Up (2+, 0-); New rule!

TS = Unchanged (1+, 1-)

TS = Unchanged, DJI = Up (1+, 0-); New rule!

\noindent
Sell:

DJI = Unchanged (1+, 1-)

DJI = Unchanged, TS = Down (1+, 0-); New rule!

\noindent
Wait:

DJI = Up (1+, 0-); New rule!

DJI = Unchanged (1+, 0-); New rule!

\pagebreak
\mysection{2}{Task 1.2}
Entropy for any set is defined as:
\begin{equation*}
  E(S) = \sum -p(i) \log_2 p(i)
\end{equation*}

Using that I calculate entropy for Action set to be:
\begin{align*}
E(Action) &= \sum -p(a) \log_2 p(a)\\
          &= -\frac{3}{8} \log_2 \frac{3}{8} -\frac{3}{8} \log_2 \frac{3}{8} -\frac{2}{8} \log_2 \frac{2}{8}\\
          &= 0.53064 + 0.53064 + 0.50000\\
          &= \uunderline{1.5613}
\end{align*}

\mysection{3}{Task 1.3}
Information gain for a feature is defines as:
 
\begin{equation*}
Gain(S,A) = E(S) - \sum \left(\frac{\left|S_v\right|}{S} \cdot E\left(S_v\right) \right)
\end{equation*}

Using that I can calculate information gain for DJI, CR, and TS.

\begin{align*}
Gain(S,DJI) &= E(S) - \frac{4}{8}E(S,Up) - \frac{2}{8}E(S,Down) - \frac{2}{8}E(S,Unchanged) \\
            &= 1.5613 - \frac{1}{2}0.81128 - \frac{1}{4}0 - \frac{1}{4}1 \\
            &= \uunderline{0.90564}\\
E(S,Up) &= -\frac{3}{4}\log_2\frac{3}{4} -\frac{1}{4}\log_2\frac{1}{4} -0\log_20\\
        &= 0.31128 + 0.50000 + 0 = \underline{0.81128}\\
E(S,Down) &= -\frac{2}{2}\log_2\frac{2}{2} -0\log_20 -0\log_20\\
          &= 0 + 0 + 0 = \underline{0}\\
E(S,Unchanged) &= -\frac{1}{2}\log_2\frac{1}{2} -\frac{1}{2}\log_2\frac{1}{2} -0\log_20\\
               &= 0.50000 + 0.50000 + 0 = \underline{1.0000}
\end{align*}

\begin{align*}
Gain(S,CR) &= E(S) - \frac{4}{8}E(S,High) - \frac{4}{8}E(S,Low)\\
           &= 1.5613 - \frac{1}{2}\frac{3}{2} - \frac{1}{2}\frac{3}{2}\\
           &= \uunderline{0.061278}\\
E(S,High) &= -\frac{2}{4}\log_2\frac{2}{4} -\frac{1}{4}\log_2\frac{1}{4} -\frac{1}{4}\log_2\frac{1}{4}\\
          &= 0.50000 + 0.50000 + 0.50000 = \underline{1.5000}\\
E(S,Low) &= -\frac{2}{4}\log_2\frac{2}{4} -\frac{1}{4}\log_2\frac{1}{4} -\frac{1}{4}\log_2\frac{1}{4}\\
         &= 0.50000 + 0.50000 + 0.50000 = \underline{1.5000}\\
\end{align*}

\begin{align*}
Gain(S,TS) &= E(S) - \frac{4}{8}E(S,Down) - \frac{2}{8}E(S,Up) - \frac{2}{8}E(S,Unchanged) \\
           &= 1.5613 - \frac{1}{2}0.81128 - \frac{1}{4}0 - \frac{1}{4}1 \\
           &= \uunderline{0.90564}\\
E(S,Down) &= -\frac{3}{4}\log_2\frac{3}{4} -\frac{1}{4}\log_2\frac{1}{4} -0\log_20\\
          &= 0.31128 + 0.50000 + 0 = \underline{0.81128}\\
E(S,Up) &= -\frac{2}{2}\log_2\frac{2}{2} -0\log_20 -0\log_20\\
        &= 0 + 0 + 0 = \underline{0}\\
E(S,Unchanged) &= -\frac{1}{2}\log_2\frac{1}{2} -\frac{1}{2}\log_2\frac{1}{2} -0\log_20\\
               &= 0.50000 + 0.50000 + 0 = \underline{1.0000}
\end{align*}

Both DJI and TS have the same value for information gain, so one of them will be chosen to be the first layer of the decision tree that we will build for using the ID3 algorithm
\end{document} 